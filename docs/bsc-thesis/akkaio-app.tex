%
%   Copyright 2013 Katarzyna Szawan <kat.szwn@gmail.com>
%       and Michał Rus <m@michalrus.com>
%
%   Licensed under the Apache License, Version 2.0 (the "License");
%   you may not use this file except in compliance with the License.
%   You may obtain a copy of the License at
%
%       http://www.apache.org/licenses/LICENSE-2.0
%
%   Unless required by applicable law or agreed to in writing, software
%   distributed under the License is distributed on an "AS IS" BASIS,
%   WITHOUT WARRANTIES OR CONDITIONS OF ANY KIND, either express or implied.
%   See the License for the specific language governing permissions and
%   limitations under the License.
%

\section{Akka.io application}
\label{sec:akka-app}
Akka component was implemented almost entirely according to our project. In order to make it possible to communicate between Android devices and actor system on the server-side there are several REST web services from Spray.io. When more than one user is working on the mind map, each device gets its own actor.  Two-directional communication is implemented by long-polling: a mobile app initiates a connection with a REST service which waits with responding until its actor receives a message from another actor. See \cref{subsec:android-akka-comm} for theoretical details.  

\subsection{Actors system }
\label{subsection:akka-actors}
The actor model is described in \cref{subsection:akka-actors}. `Since Akka enforces parental supervision, every actor is supervised and (potentially) the supervisor of its children'~\cite{AkkaDoc:2013:Actors}. In order to receive messages, every actor must implement a \inlinecode{receive} method, which describes its initial behavior.

In our application we have a number of actors which perform various functions.

\todo{\michal{ocb z tym listingiem?}}\inlinecode{Poller} 
\inlinecode{Updater}
\inlinecode{Service}

\subsection{Database and Squeryl}
\label{subsection:akka-database}

\subsection{Spray and JSON}
\label{subsection:akka-spray}
Spray is an additional layer which enables connections between Akka and client applications using JSON format messages.

In our application we added a trait \inlinecode{CustomJsonFormats}, which does necessary additions (type-safe UUID conversion) to the default JSON protocol provided by Spray.

\subsection{Synchronization}
\label{subsection:akka-synchro}
The synchronization is based \emph{only} on the server time, as planned. Client times could be unsynchronized with atomic time (and most probably would be) which renders them undependable.

There are two API URLs:

\begin{verbatim}
private def urlForPoll(since: Long) = s"$baseUrl/poll/since/$since"
private def urlForUpdate = s"$baseUrl/update"
\end{verbatim}

Every node has its cloudTime written in a database. The cloudTime is database time of the node's last modification.

\todo{\michal{blabla, nie rozumiem}}Server does not have to keep the state of the connected devices. Every device checks the highest time of modification and sends a request for update since that time.


\begin{figure}[h]
	\centering
	\missingfigure{\kasia{2 devices, collaboration} \michal{ale jak to ma wyglądać? to samo na obu?}}
	\caption{View of collaboration.}
	\label{fig:screen-collaboration}
\end{figure}
