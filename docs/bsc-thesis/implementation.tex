%
%   Copyright 2013 Katarzyna Szawan <kat.szwn@gmail.com>
%       and Michał Rus <m@michalrus.com>
%
%   Licensed under the Apache License, Version 2.0 (the "License");
%   you may not use this file except in compliance with the License.
%   You may obtain a copy of the License at
%
%       http://www.apache.org/licenses/LICENSE-2.0
%
%   Unless required by applicable law or agreed to in writing, software
%   distributed under the License is distributed on an "AS IS" BASIS,
%   WITHOUT WARRANTIES OR CONDITIONS OF ANY KIND, either express or implied.
%   See the License for the specific language governing permissions and
%   limitations under the License.
%

\chapter{Implementation}
\label{chap:implementation}

\section{Introduction}
\label{sec:impl-intro}

\subsection{Choosing IDE}
\label{subsec:choosing-ide}
We started with setting up a development environment. Since our language of choice is Scala, it was important to find an IDE which provides support for both Android and Scala development and can integrate these two elements. Quickly it turned out that there is no out-of-the-box solution. Three most popular development environments are Ecplise, NetBeans and IntelliJ, and all of them support Scala development. 

After some research and because of previous experience with Android development, we decided to work in IntelliJ IDEA 12 (and we switched to 13 as soon as it was released). We chose this particular solution because it provides seperate module facets for Android and Scala projects\cite{Steingress:2011:AndroidScala}. Also, it seems that it is the most advanced tool when it comes to code assistance. One of its most helpful features is smart code completion for both Scala and Android-specific files\cite{Steingress:2011:AndroidScala}. Also, IntelliJ IDEA offers a great deal of code analysis tools, which help to locate possible bugs, dead code and performance issues. The whole process of compiling a project is sped up by external build -- it means that all compilation tasks are run as a process separated from other IDEA processes. Apart from speeding compilation up, it decreases IDE's memory consumption\cite{Fatin:2012:NewWay}. The latest version (13)  has built-in SBT (Simple Build Tool) integration -- this term will be explained in the next section. 

\subsection{Is developing Android in Scala a good idea?}
\label{subsec:good-idea}
Developing Android applications in Scala has recently become more and more popular. However, there is a number of possible problems that developer can meet. One of the most important drawbacks is that the building is significantly slower than regular Java application. It is due to the fact, that Scala libraries need to be converted to Dalvik (Android's virtual machine) bytecode during the build process. There are many ways to make building faster, for example using ProGuard. It shrinks and optimizes code, thus making it smaller\cite{Berkel:2011:preinstall}. Another way is to preload Scala dexed (JVM bytecode translated to Dalvik bytecode) libraries on the Android device and add them to the Android runtime, which eliminates the need to do it during the build process\cite{Berkel:2011:preinstall}. However, it requires access to root account on the device, which is usually not provided as preset. Moreover, rooting most Android devices involves certain actions which can lead to voiding a warranty. 

\subsection{Android integration with SBT}
\label{subsec:choosing-ide}
Since, as mentioned, there is no out-of-the-box solution for Android development in Scala, we had to find an SBT plugin which integrates both modules and makes the build process as fast as possible. SBT is an open source build tool for Scala and Android projects similar to Maven or Ant. It is used to invoke Scalac compiler, which compiles the source code to the JVM bytecode. One of the main reasons why it is so popular is that it provides an interactive shell -- REPL, from which one can access all classes of the project. Another is incremental compiling, thanks to this feature only modified files are recompiled. \cite{Fatin:2012:NewWay}. A way to extend SBT's functionality is using one of many available plugins, or writing one.

At first we used an android sbt plugin written by Walter Chang, Mark Harrah and Jan Berkel, available on github.com as "android-plugin" repository. 
	
Due to a few reasons we decided to switch to another SBT plugin written by Perry Nguyen, also available on github.com as "android-sbt-plugin" repository. It turned out to be much quicker, as it does not require preloading 

\section{Android application}
\label{sec:android-app}

\subsection{Navigation}
\label{subsec:drawing}
The first step in creating an Android application was writing the basic views. According to the project, our application should consist of two "views" -- the first with list of mind maps and the second, with a view of a single map. Both views share an action bar with action buttons(for example: open a file, close a map, add a new map) and application's name and icon, as well as a navigation bar with tabs. 

\subsubsection{Action Bar}
\label{subsubsec:action-bar}
A basic component of most Android applications is an Action Bar. It is one of the most important design elements due to the fact, that most Android applications share it. Due to this consistency, a potential user is instantly familiar with application's basic interface. 

There is a number of things that can be put inside Action Bar, for example search bar, action buttons etc. In our application we added the following components:
\begin{itemize}
	\item Button for importing an existing mind map  from an XMind file (in a main view with list of available maps)
	\item Button for adding a new mind map (in a main view with list of available maps)
	\item Button for closing an opened mind map (in a single map view)
	\item Navigation bar which consists of tabs with opened maps (default tab is a view with list of available maps)
\end{itemize}

We want our application to be runnable even on older Android version, so instead of regular ActionBar libraries we decided to use support libraries, which take care of backward compatibility from 2.x Android versions.

\inlinecode{ActionBarSherlock} is an standalone library designed to simplify the use of the action bar in all versions of Android through a single API. The library will automatically use the native \inlinecode{ActionBar} implementation on Android 4.0 or later\cite{Wharton:2013:sherlock}. Using it is fairly simple -- the main difference is in \inlinecode{ActionBar} classes' and methods' names -- most of them are prefixed with Sherlock (for example: \inlinecode{SherlockActivity, SherlockFragmentActivity, SherlockFragmen} etc.), the same with imports. 

\subsubsection{Multi-directional ScrollView}
\label{subsubsec:action-bar}
Our application' most important feature -- creating mind maps -- requires a convenient way of navigating through a map. It should support multi-directional scrolling. In Android API there is no class which can handle multi directional scrolling. Some classes, like \inlinecode{TextView} or \inlinecode{ListView} have vertical scrolling implemented, but generally, scrolling is handled by two classes extending \inlinecode{android.widget.FrameLayout} -- \inlinecode{HorizontalScrollView} and \inlinecode{ScrollView}. 

In our application we implemented multi-directional scrolling in a class \inlinecode{Horizontal ScrollViewWithPropagation} which extends \inlinecode{HorizontalScrollView}. Inside, there is a \inlinecode{ScrollView} variable. If this inner \inlinecode{ScrollView} is set, we obtain o copy of \inlinecode{MotionEvent}, then we transform this copy in X axis and dispatch to the inner \inlinecode{ScrollView}. In the end, the original \inlinecode{MotionEvent} is handled in \inlinecode{Horizontal ScrollViewWithPropagation}.

\begin{figure}[h]
	\centering
	\missingfigure{\michal{List view screen}}
	\caption{View of mind map list, initial screen.}
	\label{fig:screen-maplist}
\end{figure}

\subsection{Drawing mind maps}
\label{subsec:drawing}
In the next stage of work we focused on a single mind map view. The main components used in drawing are \inlinecode{NodeView} representing a single mind node and \inlinecode{ArrowView} which serves as a line connecting mind nodes. They are placed in the \inlinecode{MindFragment}. Every \inlinecode{NodeView}  is paired with SubtreeWrapper, which acts as a "container" circumjacent mind node and its children.

\inlinecode{NodeView} is implemented as extending \inlinecode{FrameLayout}. It extends a \inlinecode{ViewGroup}, and we chose it because mind node's view have two children: a text field with node's content and a button. \inlinecode{FrameLayout} can be used as a parent for multiple children and it can control their position within the \inlinecode{ViewGroup} by assigning gravity to each child with the android:layout\_gravity attribute. Child views are drawn in a stack, with the latest added child on top \cite{API:2013:fl}.

Arrow is implemented as a single View. It connects a mind node with its parent. If a mind node has multiple children, then lines coming out of the child nodes are connected in one point and then a single line goes to the parent node. It makes the whole map more readable. 

In order to keep the code readability, we moved he process of drawing to a separate class -- \inlinecode{MapPainter}, whose constructor is called in the \inlinecode{onDrawView} method of the \inlinecode{MapFragment}. This class provides all necessary values and methods needed to draw a mind map:
\begin{itemize}
	\item paper padding and  subtree margin
	\item the horizontal distance between child nodes
	\item \inlinecode{arcShortRadius} which determines the radius depending on the number of children which is used later in map painter to calculate the position of mind node on the X axis
	\item \inlinecode{nodeViewSize} setting the size of a mind node
	\item \inlinecode{initializeNodeV} which initializes inflated mind node layout  
	\item \inlinecode{updateNodeView} setting the content of a mind node
\end{itemize}
 As soon as other elements are prepared --  \inlinecode{HorizontalScrollView} and its inner \inlinecode{ScrollView} layouts are found and set, current mind map is found by uuid and the paper layout is set -- a map painter is called. 

When it comes to the positioning, it was one of the most difficult tasks we faced. \inlinecode{MapPainter} handles calculating the position of mind nodes and arrows. It also manages \inlinecode{SubtreeWrappers} and their accordance with \inlinecode{NodeViews}. It can update the size and coordinates of mind nodes, as well as remove them, or recalculate nodes' positions. 

\subsection{Creating, editing and removing mind nodes}
\label{subsec:drawing}



\begin{figure}[h]
	\centering
	\missingfigure{\michal{Mind map view screen}}
	\caption{View of mind map.}
	\label{fig:screen-map}
\end{figure}

\subsection{Importing from existing .xmind files}
\label{subsec:import}

\begin{figure}[h]
	\centering
	\missingfigure{\kasia{A file chooser screen}}
	\caption{View of file chooser - importing an existing .xmind file.}
	\label{fig:screen-filechooser}
\end{figure}


\subsection{User interface}
\label{subsec:ui}
Thanks to multi-directional scrolling it is easy to navigate even if the map is big.

\todo[inline]{\kasia{Why we designed UI in such a way.}}
\section{Akka.io application}
\label{sec:akka-app}
Akka component was implemented almost entirely according to our project. In order to make it possible to communicate between Android devices and actor system on the server-side there are several REST web services from Spray.io. When more than one user is working on the mind map, each device gets its own actor.  Two-directional communication is implemented by long-polling: a mobile app initiates a connection with a REST service which waits with responding until its actor receives a message from another actor. See \cref{subsec:android-akka-comm} for theoretical details.  

\subsection{Actors system }
\label{subsection:akka-actors}
The actor model is described in \cref{subsection:akka-actors}. "Since Akka enforces parental supervision every actor is supervised and (potentially) the supervisor of its children" \cite{AkkaDoc:2013:Actors}. In order to receive messages, every actor must implement \inlinecode{receive} methods, which describe its behavior. 

In our application we have a number of actors which perform various functions.
\inlinecode{Poller} 
\inlinecode{Updater}
\inlinecode{Service}

\subsubsection{Supervisor actor}
\label{subsubsection:akka-actors-supervisor}
Main supervisor actor is provided by Akka; tightly integrated with the actor system. There is always one such actor for each system.
	
\subsubsection{Per-user actors}
\label{subsubsection:akka-actors-peruser}

\subsection{Database and Squeryl}
\label{subsection:akka-database}

\subsection{Spray and JSON}
\label{subsection:akka-spray}
Spray is an additional layer which enables connection between Akka and application using JSON protocol. 

In our application we added a trait \inlinecode{CustomJsonFormats}, which does necessary modifications in default JSON protocol. 

\subsection{Synchronization}
\label{subsection:akka-synchro}
The synchronization is based on the server time. 
There are two API URLs: 
\begin{verbatim}
private def urlForPoll(since: Long) = s"$baseUrl/poll/since/$since"
private def urlForUpdate = s"$baseUrl/update"
\end{verbatim}
Every node has written its cloudTime (meaning server time). Server does not have to keep the state of the connected devices -- every device checks the highest time of modification and sends a request for update since that time.


\begin{figure}[h]
	\centering
	\missingfigure{\michal{2 devices, collaboration}}
	\caption{View of collaboration.}
	\label{fig:screen-collaboration}
\end{figure}

%
%   Copyright 2013 Katarzyna Szawan <kat.szwn@gmail.com>
%       and Michał Rus <m@michalrus.com>
%
%   Licensed under the Apache License, Version 2.0 (the "License");
%   you may not use this file except in compliance with the License.
%   You may obtain a copy of the License at
%
%       http://www.apache.org/licenses/LICENSE-2.0
%
%   Unless required by applicable law or agreed to in writing, software
%   distributed under the License is distributed on an "AS IS" BASIS,
%   WITHOUT WARRANTIES OR CONDITIONS OF ANY KIND, either express or implied.
%   See the License for the specific language governing permissions and
%   limitations under the License.
%

\section{Encountered problems and their solutions}
\label{sec:impl-problems}

\subsection{Encapsulation of bidirectional message passing over request-response style HTTP protocol}
\label{subsec:problem-longpolling}

Most web browsers have a limit for concurrently open connections to the same server set to a value close to 2. This means it would be best to encapsulate message passing in \emph{one} constantly open connection, to leave the other free to be used in any way needed (e.g. to download some media resources).

This is not much of an issue when Android is concerned, because as many connections can be opened as we need. However, as it is really easy to add other front-ends (web application), it is wise to implement the REST actors (\cref{subsection:akka-actors}) and the client to be able to operate sequentially on one connection only.

What has to be done on the client-side is to disconnect/destroy the connection with a request og \inlinecode{GET /poll/since/\$TIME} and only then send \inlinecode{POST /update}. This is almost equally straightforward with Spray: we have to overload \inlinecode{onConnectionClosed} method of \inlinecode{HttpServiceBase} and in its new, overloaded body stop an appriopriate Poller actor.

The final choice whether to use one or two connections is left to the client: to our Akka backend both of these cases are of completely no difference.

\subsection{No removing of tabs in Android's TabHost}
\label{subsec:problem-tabhost}

Strange as it may seem, the system-provided TabHost component does not allow a removal of tabs on the same level of abstraction as it allows adding them. To remove a tab, one has to find the tab's \emph{raw} view and remove the view from its parent, the TabHost. On the other hand, to add a new tab, one has to create a TabSpec object and add this object to the TabHost. Quite a discrepancy.

It has been decided to keep track of all created TabSpecs manually and at removal request, remove \emph{all} of added tabs, remove a corresponding TabSpec from our collection and then recreate all tabs from this just reduced collection.

\subsection{No bidirectional scroll view in Android standard components}
\label{subsec:problem-scrollview}
In our application we implemented multi-directional scrolling in a class \inlinecode{Horizontal ScrollViewWithPropagation} which extends \inlinecode{HorizontalScrollView}. Inside, there is a \inlinecode{ScrollView} variable. If this inner \inlinecode{ScrollView} is set, we obtain o copy of \inlinecode{MotionEvent}, then we transform this copy in X axis and dispatch to the inner \inlinecode{ScrollView}. In the end, the original \inlinecode{MotionEvent} is handled in \inlinecode{Horizontal ScrollViewWithPropagation}.

\subsection{Positioning child nodes}
\label{subsec:problem-positioning}
\todo[inline]{\kasia{M., describe positioning nodes.}}

