% przykład pracy w języku angielskim

\documentclass[american]{inz}
\usepackage[OT4]{fontenc}
\usepackage[latin2]{inputenc}

%\title{Miły tytuł pracy}
%\author{Alojzy Bąbel}
%\date{2007}
%\advisor{dr Jan Promotor}


% \documentclass[]{inz}
% \usepackage[OT4]{fontenc}
% \usepackage[latin2]{inputenc}

\title{A nice title}
% W przypadku pracy w jęz. angielskim należy również podac tytył po polsku:
\titlepl{Miły tytuł polski}

% język pracy zdefiniowany jest opcją klasy mgr, patrz 1-sza linia: 
% \documentclass
% obslugiwane jezyki: polish, american, english
\author{Alojzy B±bel}
\date{2012}
\advisor{Jan Promotor, Ph.D.}
% % oraz promotor po polsku
\advisorpl{dr Jan Promotor}


\begin{document}

\maketitle


\chapter{Wstęp}
\label{chap:wstep}

Czyli o tym jak się wszystko zaczęło.
Czyli o tym jak się wszystko zaczęło.
Czyli o tym jak się wszystko zaczęło.
Czyli o tym jak się wszystko zaczęło.
Czyli o tym jak się wszystko zaczęło.
Czyli o tym jak się wszystko zaczęło.
Czyli o tym jak się wszystko zaczęło.
Czyli o tym jak się wszystko zaczęło.
Czyli o tym jak się wszystko zaczęło.
Czyli o tym jak się wszystko zaczęło.
Czyli o tym jak się wszystko zaczęło.
Czyli o tym jak się wszystko zaczęło.
 

Czyli o tym jak się wszystko zaczęło.
Czyli o tym jak się wszystko zaczęło.
Czyli o tym jak się wszystko zaczęło.

\section{Postawienie Problemu}
\label{sec:problem}

\section{Główny Cel}
\label{sec:cel}


\chapter{Podstawy Teoretyczne}
\label{chap:teoria}



\section{Opis zagadnień 1}
\label{sec:zagadnienia1}

\section{Opis zagadnień 2}
\label{sec:zagadnienia2}


\chapter{Projekt}
\label{chap:projekt}


\section{Wymagania}
\label{sec:wymagania}


\section{Projekt rozwi±zania}
\label{sec:rozwiazanie}



\chapter{Realizacja}
\label{chap:realizacja}


\section{Implementacja}
\label{sec:implementacja}



\chapter{Podsumowanie}
\label{chap:podsumowanie}


\bibliographystyle{plain}
%plik z baza danych bibliograficznych w formacie BibTeX'a (bibliografia.bib)
\bibliography{bibliografia}

\appendix

\chapter{Dodatek}
\label{chap:dodatek}



\end{document}

