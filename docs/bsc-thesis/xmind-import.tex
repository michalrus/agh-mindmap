%
%   Copyright 2013 Katarzyna Szawan <kat.szwn@gmail.com>
%       and Michał Rus <m@michalrus.com>
%
%   Licensed under the Apache License, Version 2.0 (the "License");
%   you may not use this file except in compliance with the License.
%   You may obtain a copy of the License at
%
%       http://www.apache.org/licenses/LICENSE-2.0
%
%   Unless required by applicable law or agreed to in writing, software
%   distributed under the License is distributed on an "AS IS" BASIS,
%   WITHOUT WARRANTIES OR CONDITIONS OF ANY KIND, either express or implied.
%   See the License for the specific language governing permissions and
%   limitations under the License.
%

\subsection{XMind import}
\label{subsec:xmind-exchange}

XMind files (`workbooks') are saved as a ZIP archive of mostly \inlinecode{.xml} files, two of which are the most important and are always present. First, \inlinecode{content.xml} stores data and its hierarchy, and the second, \inlinecode{META-INF/manifest.xml} is the list of files included in the archive. An {\em XMind} file could also contain separate \inlinecode{.xml} documents for content and styles, a \inlinecode{.jpg} image file for thumbnails, and directories for related attachments. \cite{XMind:2009:Format}

\todo[inline]{\michal{M./K., finish XMind import: write about the recursive import method.}}
