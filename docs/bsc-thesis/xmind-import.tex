%
%   Copyright 2013 Katarzyna Szawan <kat.szwn@gmail.com>
%       and Michał Rus <m@michalrus.com>
%
%   Licensed under the Apache License, Version 2.0 (the "License");
%   you may not use this file except in compliance with the License.
%   You may obtain a copy of the License at
%
%       http://www.apache.org/licenses/LICENSE-2.0
%
%   Unless required by applicable law or agreed to in writing, software
%   distributed under the License is distributed on an "AS IS" BASIS,
%   WITHOUT WARRANTIES OR CONDITIONS OF ANY KIND, either express or implied.
%   See the License for the specific language governing permissions and
%   limitations under the License.
%

\subsection{XMind import and export}
\label{subsec:xmind-exchange}

Details of an XMind file format can be found in \cref{sec:xmind}. XML is a way of storing data as a labeled tree created of nested tags with various attributes. Most of the attributes in  XMind tags will not be necessary, and some of them are not even used in XMind. Specifically, we want to omit data that is used to determine a style assigned to the sheet. The first step to importing an {\em XMind} file is to open it as a ZIP archive. The inlinecode{content.xml} is wrapped in an \inlinecode{InputStream}, and converted to Scala's XML object. Then, the actual parsing may be started.

Having XML support on the language level, Scala offers a very concise syntax for dealing with it~\cite{Odersky:2008:Programming}. It allows a convenient navigation through the data using \inlinecode{\textbackslash} and \inlinecode{\textbackslash\textbackslash} operators, meaning, respectively, `search in direct children' and `search in all subtags.' A content of a tag can be then obtained by using the \inlinecode{.text} method.

The \inlinecode{content.xml} file may consist of a lot of sheets (represented by \inlinecode{<sheet/>} tag with \inlinecode{id} attribute), so parsing should be done within a functional \emph{map} operation, resulting in the same number of mind maps as the number of sheets. For each sheet found, a new \inlinecode{MindMap} object is created. Then a content of the root node is set by finding sheet's first child (sheet's children are represented as \inlinecode{<topic/>}'s) and reading the content of its child, a \inlinecode{<title/>}. A single topic represents a node, which is then saved as a \inlinecode{MindNode} object. Root's children nodes are then imported recursively, using exactly the same method.

When exporting, these actions have to be reversed. First, the \inlinecode{content.xml} file is synthesized according to the rules governing the XMind format. Next, it is zipped to a file with an \inlinecode{.xmind} extension and placed on user's SD card. It's worth noting that only one file in this ZIP archive---\inlinecode{content.xml}---is absolutely sufficient to make the map readable to original XMind application.
